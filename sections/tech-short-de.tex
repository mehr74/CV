\documentclass[../main.tex]{subfiles}
\begin{document}
%- experiences ------------------------------------------------------------%
  \pdfbookmark{Berufserfahung}{Berufserfahung}
  \begin{category}
    \section{Berufserfahung}
    \citembullet{
      \textbf{Softwareentwickler} bei \textbf{Deutsche Börse} 
      \strut\hfill Februar 2023 -- Juli 2023 \\[-9pt]
    }
      \begin{itemize}
        \item \emph{\textbf{Projekt:} Cloud Stream}
        Entwickelte und pflegte einen optimierten Client, um Nachrichten mit 
        höchster Geschwindigkeit zu erfassen und Stresstests für das 
        Framework durchzuführen. Verbesserte die bestehende Architektur 
        durch die Einführung statischer Images, die mit Packer erstellt 
        wurden, und half dem Team bei der Migration des Produkts von
        Amazon Web Services zu Google Cloud Platform.
      \end{itemize}

    \citembullet{
      \textbf{Wissenschaftlicher Mitarbeiter} bei 
      \textbf{Max Planck Insitut für Softwaresystemme} 
      betreut von 
      \textbf{\href{https://people.mpi-sws.org/~antoinek}
      {Prof. Antoine Kaufmann}}
    } \strut\hfill Oktober 2019 -- Januar 2023\\[-9pt]
      \begin{itemize}
        \item \emph{\textbf{Project:} 
          TCP-Beschleunigung als Service in virtualisierten Umgebungen
        }
        Erforschte den Anforderungen an die Netzwerkvirtualisierung 
        in einer öffentlichen Cloud-Umgebung. Entwickelte eine Lösung, 
        die moderne, CPU-effiziente Netzwerkstacks für Kunden in der 
        Cloud ermöglicht. Ein Netdev-Treiber wurde dem Open vSwitch 
        Code hinzugefügt und TAS als effizienter Paketverarbeitungsstapel 
        für TCP wurde verwendet.


        \item \emph{\textbf{Project:} 
        Erforschung bereichsspezifischer Architekturen für die 
        Verarbeitung von Netzwerkprotokollen 
        } 
        Arbeitet mit Xilinx UltraScale+ FPGA-basierten NICs. 
        Implementiert einen Userspace-Treiber für die PCIe-Kommunikation 
        mit Hilfe des vfio-Treibers

      \end{itemize}

    \citembullet{
      \textbf{Forschungspraktikant} at \textbf{SAFARI Group} 
      betreut von \textbf{
        \href{https://people.inf.ethz.ch/omutlu/}{Prof. Onur Mutlu}
      }, ETH  \hfill Juli - September 2018
      \begin{itemize}
        \item \emph{\textbf{Project:} 
        Richtung praktischer und effizienter 
        Hardware-Software-Schnittstellen 
      } 

      Untersuchte die Möglichkeit, reichhaltige 
      Hardware-Software-Schnittstellen zu entwerfen. Fügte benutzerdefinierte 
      Anweisungen zu RISC-V hinzu und implementierte ein benutzerdefiniertes 
      Modul auf Rocket Chip, das von Programmierern verwendet wird, um 
      Datenstrukturen auf höherer Ebene an die zugrunde liegende Hardware 
      zu übermitteln.
    \end{itemize}
    }

    \citembullet{
      \textbf{HiWi} 
      bei \textbf{
        \href{http://dsn.ce.sharif.edu/}
        {Data Storage, Networks, \& Processing}
      }
      betreut von
      \textbf{
        \href{http://sharif.edu/~asadi}{Prof. Hossein Asadi}
      }, Sharif University of Technology  \hfill 
      September 2016 - April 2018 \\[-9pt]
    \begin{itemize}
       \item \emph{\textbf{Project:} SSD-Zuverlässigkeit unter ungünstigen Bedingungen }

       Entwickelte ich einen Rahmen für automatische Fehlertests, 
       der aus Hardware- und Softwareabschnitten besteht. 
       Durch die Anwendung des Test-Frameworks habe ich mehr 
       als 5 SSDs von verschiedenen Herstellern getestet und 
       verschiedene Arten von Daten- und Gerätefehlern erkannt.

    \end{itemize}
  }
\end{category}
\end{document}
